\documentclass{beamer}
\usepackage{ctex}

% 样式
\usetheme{Copenhagen}
\usecolortheme[RGB={0,86,32}]{structure}
\setbeamertemplate{headline}{}
\setbeamertemplate{itemize items}[circle]
\setbeamertemplate{enumerate items}[default]
\setbeamertemplate{sections/subsections in toc}[square]

% 图片
\graphicspath{{img/}}

% 公式
\usepackage{amsmath,bm}
\usepackage{lmodern}
\usepackage{utopia}
\usepackage{amsfonts}
\usepackage{amsthm}
\usepackage{amssymb}

% 字体
\setCJKmainfont[BoldFont={STHeiti}, ItalicFont={STKaiti}]{STSong}
\setCJKsansfont{PingFang SC}
\setCJKmonofont{PingFang SC}
\RequirePackage{fontspec,xltxtra,xunicode}
\setmainfont[Mapping=tex-text]{Times New Roman}
\setsansfont[Mapping=tex-text]{Helvetica Neue}
\setmonofont{Consolas}
\usefonttheme[onlymath]{serif}

% 模块
\usepackage{booktabs}
\usepackage{multirow}
\usepackage{multicol}
\usepackage{xcolor}
\usepackage{listings}
\usepackage{tasks}
\usepackage{tabto}
\usepackage[absolute,overlay]{textpos}
\usepackage[normalem]{ulem}
\newcommand\redout{\bgroup\markoverwith
{\textcolor{red}{\rule[0.5ex]{2.5pt}{0.8pt}}}\ULon}

% 代码
\definecolor{codered}{rgb}{0.8,0,0}
\definecolor{codegreen}{rgb}{0,0.6,0}
\definecolor{codeblue}{rgb}{0,0,0.9}
\definecolor{codegray}{rgb}{0.5,0.5,0.5}
\definecolor{codepurple}{rgb}{0.58,0,0.82}
\definecolor{backcolour}{rgb}{0.95,0.95,0.92}
\lstdefinestyle{vscode}{
    % backgroundcolor=\color{backcolour},
    commentstyle=\ttfamily\color{codegreen},
    keywordstyle=\color{codeblue},
    numberstyle=\scriptsize\color{codegray},
    stringstyle=\color{codered},
    basicstyle=\ttfamily\footnotesize,
    breakatwhitespace=false,
    breaklines=true,
    captionpos=b,
    keepspaces=true,
    numbers=left,
    numbersep=5pt,
    showspaces=false,
    showstringspaces=false,
    showtabs=false,
    tabsize=2
}
\lstset{style=vscode}
\lstset{inputpath={code/}}

% 流程图
\usepackage{tikz}
\usetikzlibrary{shapes,arrows}
\tikzstyle{startstop} = [rectangle, rounded corners, minimum width=1cm, minimum height=.5cm, text centered, text width=1.5cm, draw=black, font=\footnotesize]
\tikzstyle{io} = [trapezium, trapezium left angle=70, trapezium right angle=110, minimum width=1cm, minimum height=.5cm, text centered, text width=1.5cm, draw=black, font=\footnotesize]
\tikzstyle{process} = [rectangle, minimum width=1cm, minimum height=.5cm, text centered, text width=1.5cm, draw=black, font=\footnotesize]
\tikzstyle{decision} = [diamond, aspect=3, text centered, draw=black, font=\footnotesize]
\tikzstyle{arrow} = [thin,->,>=stealth, font=\footnotesize]
\usetikzlibrary{
    arrows.meta,
    tikzmark,
    calc,
    positioning
}
\usetikzmarklibrary{listings}
\newcommand{\redbox}[5]{
    \draw[red, very thick] ([shift={(#3*4.78pt-6pt, .66em)}] pic cs:line-#1-#2-start) rectangle ([shift={(#5*4.78pt+2pt, -.24em)}] pic cs:line-#1-#4-start)
}
\usetikzlibrary{matrix}
\usetikzlibrary{overlay-beamer-styles}

% 段落与缩进
\setlength{\leftmargini}{12pt}
\usepackage{xpatch}
\xpatchcmd{\itemize}
  {\def\makelabel}
  {\ifnum\@itemdepth=1\relax
     \setlength\itemsep{1.5ex}% separation for first level
     \lstset{basicstyle=\ttfamily\small}
   \else
     \ifnum\@itemdepth=2\relax
       \setlength\itemsep{1ex}% separation for second level
       \setlength\topsep{1ex}% separation for second level
     \else
       \ifnum\@itemdepth=3\relax
         \setlength\itemsep{1ex}% separation for third level
         \setlength\topsep{1ex}% separation for third level
         \lstset{basicstyle=\ttfamily\scriptsize}
   \fi\fi\fi\def\makelabel
  }
 {}
 {}
\xpatchcmd{\beamer@enum@}
  {\def\makelabel}
  {\ifnum\@itemdepth=1\relax
     \setlength\itemsep{1.5ex}% separation for first level
     \lstset{basicstyle=\ttfamily\small}
   \else
     \ifnum\@itemdepth=2\relax
       \setlength\itemsep{1ex}% separation for second level
       \setlength\topsep{1ex}% separation for second level
     \else
       \ifnum\@itemdepth=3\relax
         \setlength\itemsep{1ex}% separation for third level
         \setlength\topsep{1ex}% separation for third level
         \lstset{basicstyle=\ttfamily\scriptsize}
   \fi\fi\fi\def\makelabel
  }
 {}
 {}

%------------------------------------------------------------
\AtBeginSection[]
{
    \begin{frame}
        \frametitle{目录}
        \tableofcontents[currentsection,hideallsubsections]
    \end{frame}
}
%------------------------------------------------------------


%------------------------------------------------------------
\title[Sicily 信息学公益课]
{Sicily 信息学公益课}

\subtitle{课程介绍}

\author[Beiyu Li]
{Beiyu Li\\
\texttt{<sysulby@gmail.com>}}

% \institute[SOJ]
% {Sicily Online Judge}

\date[\today]
{\number\year 年 \number\month 月 \number\day 日}
%------------------------------------------------------------


\begin{document}

\author[sysulby]
{SOJ 信息学竞赛教练组}

\begin{frame}
    \titlepage
\end{frame}
\setcounter{framenumber}{0} % 标题页不编号

\begin{frame}[fragile]
    \frametitle{主讲教师}

    \begin{columns}
        \column{.36\textwidth}
        \begin{figure}
            \includegraphics[width=\textwidth]{beiyuli.jpg}
        \end{figure}

        \column{.64\textwidth}
        李贝瑀

        \vspace{.5em}

        {\footnotesize 中山大学 ACM 主力队员}

        \vspace{.5em}

        {\footnotesize Codeforces \textcolor{red}{Grandmaster}}

        \vspace{.5em}

        {\footnotesize 曾就职于谷歌,多年算法竞赛参赛及命题经验}

        \vspace{.5em}

        \begin{itemize}
            \item {\footnotesize 2015 年 ACM-ICPC EC-Final 金奖}
            \item {\footnotesize 2015 年 ACM-ICPC 沈阳赛区第 5 名}
            \item {\footnotesize 2015 年 GDOI 广东省选命题组成员}
            \item {\footnotesize 2016 年 ACM-ICPC EC-Final 裁判}
        \end{itemize}
    \end{columns}
\end{frame}

\setcounter{framenumber}{0} % 介绍页不编号


\section{C++ 程序设计基础}

\subsection{课程概述}

%------------------------------------------------------------
\begin{frame}[fragile]
    \frametitle{课程概述}

    \begin{itemize}[<+->]
        \item 本阶段课程从零开始,讲授 C++ 的基础语法知识,同时涵盖部分基本的编程方法论,主要包括:

            \begin{itemize}
                \item 变量的输入输出与运算 —— 基本语句
                \item 顺序结构、分支结构与循环结构 —— 程序设计的基本结构
                \item 数组 —— 数据的存储方式
            \end{itemize}

        \item 在此基础上,强调对良好的编程风格与思维习惯的培养
    \end{itemize}
\end{frame}
%------------------------------------------------------------

\subsection{课程大纲}

%------------------------------------------------------------
\begin{frame}[fragile]
    \frametitle{课程大纲}

    \begin{multicols}{2}
        \begin{enumerate}
            \item<1-> 初识计算机与程序设计
            \item<1-> 变量与输入输出
            \item<1-> 表达式与运算
            \item<1-> 分支结构:\lstinline|if..else| 语句
            \item<1-> \textcolor{red}{阶段测试 I}
            \item<2-> 循环结构:\lstinline|while| 循环
            \item<2-> 循环结构:\lstinline|for| 循环
            \item<2-> 循环与分支综合
            \item<2-> 多重循环
            \item<2-> \textcolor{red}{阶段测试 II}
            \item<3-> 一维数组
            \item<3-> 一维数组与循环
            \item<3-> 多维数组
            \item<3-> 数组的综合应用
            \item<3-> \textcolor{red}{阶段测试 III}
        \end{enumerate}
    \end{multicols}
\end{frame}
%------------------------------------------------------------

\subsection{课程目标}

%------------------------------------------------------------
\begin{frame}[fragile]
    \frametitle{课程目标}

    \begin{itemize}[<+->]
        \item 通过该阶段课程的学习,可以掌握 C++ 的基本语法,为深入学习 C++ 的进阶知识,打下坚实的语法基础

        \item 学完该阶段课程的同学,编程能力可以达到:

            \begin{itemize}
                \item GESP C++ 一级
                \item 电子学会 C++ 一级
            \end{itemize}

    \end{itemize}
\end{frame}
%------------------------------------------------------------

\subsection{课程要求}

%------------------------------------------------------------
\begin{frame}[fragile]
    \frametitle{课程要求}

    \begin{alertblock}{前置条件}
        \begin{itemize}
            \item 具备良好的逻辑思维与数学能力

                \begin{itemize}
                    \item 无编程相关经验要求
                    \item 熟练使用键盘对学习效率有帮助,但不是必须的
                \end{itemize}

            \item 推荐年级:五年级或以上
        \end{itemize}
    \end{alertblock}
\end{frame}
%------------------------------------------------------------


\section{C++ 程序设计进阶}

\subsection{课程概述}

%------------------------------------------------------------
\begin{frame}[fragile]
    \frametitle{课程概述}

    \begin{itemize}[<+->]
        \item 本阶段课程在 C++ 的语法基础上,进一步讲授结构化的程序设计方法,主要包括:

            \begin{itemize}
                \item 进制、编码、位运算与字符串
                \item 类与对象 —— 对数据的抽象
                \item 函数与递归 —— 对过程的抽象
            \end{itemize}

        \item 在此基础上,强调对良好的编程风格与抽象思维的培养
    \end{itemize}
\end{frame}
%------------------------------------------------------------

\subsection{课程大纲}

%------------------------------------------------------------
\begin{frame}[fragile]
    \frametitle{课程大纲}

    \begin{multicols}{2}
        \begin{enumerate}
            \item<1-> 函数 I
            \item<1-> 函数 II
            \item<1-> 进制
            \item<1-> 编码与位运算
            \item<1-> \textcolor{red}{阶段测试 I}
            \item<2-> C 风格字符串
            \item<2-> \lstinline|string| 类
            \item<2-> 结构体 I
            \item<2-> 结构体 II
            \item<2-> \textcolor{red}{阶段测试 II}
            \item<3-> 内存空间与指针
            \item<3-> 递归I
            \item<3-> 递归II
            \item<3-> 递归III
            \item<3-> \textcolor{red}{阶段测试 III}
        \end{enumerate}
    \end{multicols}
\end{frame}
%------------------------------------------------------------

\subsection{课程目标}

%------------------------------------------------------------
\begin{frame}[fragile]
    \frametitle{课程目标}

    \begin{itemize}[<+->]
        \item 通过该阶段课程的学习,可以掌握编写结构化程序的方法,为进一步学习算法与数据结构知识,打下坚实的编程基础

        \item 学完该阶段课程的同学,编程能力可以达到:

            \begin{itemize}
                \item GESP C++ 二级
                \item 电子学会 C++ 二级
            \end{itemize}

    \end{itemize}
\end{frame}
%------------------------------------------------------------

\subsection{课程要求}

%------------------------------------------------------------
\begin{frame}[fragile]
    \frametitle{课程要求}

    \begin{alertblock}{前置条件}
        \begin{itemize}
            \item 完成上一阶段课程的学习

                \begin{itemize}
                    \item 或具备同等级的 C++ 语法基础
                \end{itemize}

            \item 推荐年级:五年级或以上
        \end{itemize}
    \end{alertblock}
\end{frame}
%------------------------------------------------------------


\section{C++ 算法与数据结构基础}

\subsection{课程概述}

%------------------------------------------------------------
\begin{frame}[fragile]
    \frametitle{课程概述}

    \begin{itemize}[<+->]
        \item 本阶段课程在 C++ 程序设计的基础上,讲授基本的算法与数据结构知识,主要包括:

            \begin{itemize}
                \item 算法复杂度分析
                \item 排序、查找、模拟与回溯
                \item 数论与组合数学基础
                \item 栈、队列、链表与 C++ STL (\lstinline|vector, set, map|)
            \end{itemize}

        \item 在此基础上,强调结合实际问题,培养应用性思维
    \end{itemize}
\end{frame}
%------------------------------------------------------------

\subsection{课程大纲}

%------------------------------------------------------------
\begin{frame}[fragile]
    \frametitle{课程大纲}

    \begin{multicols}{2}
        \begin{enumerate}
            \item<1-> 算法分析与枚举、模拟
            \item<1-> 基础排序算法
            \item<1-> \lstinline|sort()| 函数的应用
            \item<1-> 二分查找
            \item<1-> \textcolor{red}{阶段测试 I}
            \item<2-> \lstinline|vector|、\lstinline|set| 与 \lstinline|map|
            \item<2-> 栈及其应用
            \item<2-> 队列与优先队列
            \item<2-> 链表
            \item<2-> \textcolor{red}{阶段测试 II}
            \item<3-> 数论基础
            \item<3-> 组合数学基础
            \item<3-> 回溯 I
            \item<3-> 回溯 II
            \item<3-> \textcolor{red}{阶段测试 III}
        \end{enumerate}
    \end{multicols}
\end{frame}
%------------------------------------------------------------

\subsection{课程目标}

%------------------------------------------------------------
\begin{frame}[fragile]
    \frametitle{课程目标}

    \begin{itemize}[<+->]
        \item 通过该阶段课程的学习,可以掌握基本的算法知识,为投入信息学竞赛的学习,打下坚实的理论基础

        \item 学完该阶段课程的同学,编程能力可以达到:

            \begin{itemize}
                \item GESP C++ 三级
                \item CSP-J 第二轮三等奖
            \end{itemize}
    \end{itemize}
\end{frame}
%------------------------------------------------------------

\subsection{课程要求}

%------------------------------------------------------------
\begin{frame}[fragile]
    \frametitle{课程要求}

    \begin{alertblock}{前置条件}
        \begin{itemize}
            \item 完成上一阶段课程的学习

                \begin{itemize}
                    \item 具备同等级的 C++ 编程基础
                \end{itemize}

            \item 推荐年级:五年级或以上
        \end{itemize}
    \end{alertblock}
\end{frame}
%------------------------------------------------------------

%------------------------------------------------------------
\begin{frame}
    \begin{center}
        {\Huge Thank you!}
    \end{center}
\end{frame}
%------------------------------------------------------------

\end{document}

