%------------------------------------------------------------
\title[05 - C 风格字符串]
{05 - C 风格字符串}

\subtitle{C++ 程序设计进阶}

\author[Beiyu Li]
{Beiyu Li\\
\texttt{<sysulby@gmail.com>}}

% \institute[SOJ]
% {Sicily Online Judge}

\date[\today]
{\number\year 年 \number\month 月 \number\day 日}
%------------------------------------------------------------


\begin{document}

\author[sysulby]
{SOJ 信息学竞赛教练组}

\begin{frame}
    \titlepage
\end{frame}
\setcounter{framenumber}{0} % 标题页不编号

\section{复习回顾}

%------------------------------------------------------------
\begin{frame}[fragile]
    \frametitle{编码}

    \begin{itemize}
        \item<1-> 整数编码:符号位 + 数值位
            \begin{itemize}
                \item 原码:正数符号位为 $0$,负数为 $1$;数值位为数值的二进制
                \item 反码:正数的反码和原码相同;负数的反码将数值位取反
                \item 补码:正数的补码和原码相同:负数的补码在反码的基础上 $+ 1$
            \end{itemize}

        \item<2-> 字符编码
            \begin{itemize}
                \item 字符类型判断:\lstinline|if(ch >= 'a' && ch <= 'z')|
                \item 字符和整数转换:\lstinline|ch - 'a'|
                \item 大小写字母转换:\lstinline|ch - 'a' + 'A'|
            \end{itemize}
    \end{itemize}

\end{frame}
%------------------------------------------------------------

%------------------------------------------------------------
\begin{frame}[fragile]
    \frametitle{位运算}

    \begin{itemize}
        \item 位运算
            \begin{itemize}
                \item 按位与(\lstinline|&|):只有两个对应位都为 $1$ 时才为 $1$
                \item 按位或(\lstinline|||):只要两个对应位中有一个 $1$ 时就为 $1$
                \item 异或(\lstinline|^|):只有两个对应位不同时才为 $1$
                \item 取反(\lstinline|~|):符号位、数值位全部取反
                \item 左移(\lstinline|<<|):\lstinline|x << i| 相当于 $x * 2^i$
                \item 右移(\lstinline|>>|):\lstinline|x >> i| 相当于 $x / {2^i}$
            \end{itemize}
    \end{itemize}

\end{frame}
%------------------------------------------------------------

\section{字符串}

%------------------------------------------------------------
\begin{frame}[fragile]
    \frametitle{字符串}

    \begin{itemize}
        \item<1-> “字符串”就是文本信息
        \item<1-> 那我们什么时候接触过文本信息呢?
            \begin{itemize}
                \item \lstinline|cout << "文本";|
                \item Word 软件
                \item 搜索引擎
                \item ……
            \end{itemize}

        \item<2-> 通俗来说,字符串就是一串字符
            \begin{itemize}
                \item C 语言中的字符串 —— 以空字符 \lstinline|'\0'| 结束的字符数组,又称 C 风格字符串
                \item C++ 语言中的字符串 —— string 类
            \end{itemize}
    \end{itemize}

\end{frame}
%------------------------------------------------------------

%------------------------------------------------------------
\begin{frame}[fragile]
    \frametitle{字符数组}

    \begin{itemize}
        \item C 风格字符串的本质是字符数组,而字符数组的本质是元素为字符的数组
            \lstinputlisting[basicstyle=\ttfamily\scriptsize,language=C++,name=charArrayDeclare]{ch17/charArrayDeclare.cc}
    \end{itemize}

    \begin{tikzpicture}
            [nodes in empty cells, nodes={minimum width=1.5cm, minimum height=.7cm}, row sep=-\pgflinewidth, column sep=-\pgflinewidth]
            \matrix(A) [matrix of nodes, ampersand replacement=\&, row 1/.style={nodes={draw=none}}, nodes={draw, anchor=center}]{
                \lstinline|city[0]| \& \lstinline|city[1]| \& \lstinline|city[2]| \& \lstinline|city[3]| \& \lstinline|city[4]| \& \lstinline|city[5]| \& \lstinline|city[6]|  \\
                \lstinline|P|   \& \lstinline|a|   \& \lstinline|r|   \& \lstinline|i|   \& \lstinline|s|   \& \lstinline|  |   \& \lstinline|  |   \\
            };
     \end{tikzpicture}
\end{frame}
%------------------------------------------------------------

%------------------------------------------------------------
\begin{frame}[fragile]
    \frametitle{C 风格字符串}

    \begin{itemize}
        \item 在字符数组最后一个字符的下一个位置,加上字符串结束标志 \lstinline|'\0'|,一个字符数组就变成了 C 风格字符串
            \lstinputlisting[basicstyle=\ttfamily\scriptsize,language=C++,name=cstringDecalre]{ch17/cstringDecalre.cc}
        \item<2-> 在声明字符串时,数组的大小至少要大 $1$,如:\lstinline|char s[110];|
    \end{itemize}

    \begin{tikzpicture}
            [nodes in empty cells, nodes={minimum width=1.5cm, minimum height=.7cm}, row sep=-\pgflinewidth, column sep=-\pgflinewidth]
            \matrix(A) [matrix of nodes, ampersand replacement=\&, row 1/.style={nodes={draw=none}}, nodes={draw, anchor=center}]{
                \lstinline|city[0]| \& \lstinline|city[1]| \& \lstinline|city[2]| \& \lstinline|city[3]| \& \lstinline|city[4]| \& \lstinline|city[5]| \& \lstinline|city[6]|  \\
                \lstinline|P|   \& \lstinline|a|   \& \lstinline|r|   \& \lstinline|i|   \& \lstinline|s|   \& \lstinline|\0|   \& \lstinline|  |  \\
            };
     \end{tikzpicture}
\end{frame}
%------------------------------------------------------------

%------------------------------------------------------------
\begin{frame}[fragile]
    \frametitle{C 风格字符串}

    \begin{itemize}
        \item<1-> \lstinline|cin| 输入
            \begin{itemize}
                \item 当遇到空格、换行等空字符时会结束输入
                \item 默认从下标 $0$ 开始存储
                \lstinputlisting[basicstyle=\ttfamily\scriptsize,language=C++,name=input]{ch17/input.cc}
            \end{itemize}

        \item<3-> \lstinline|fgets| 整行输入
            \begin{itemize}
                \item 当遇到换行时会结束输入,会把空格、\textbf{换行符}全部读入
                \lstinputlisting[basicstyle=\ttfamily\scriptsize,language=C++,name=inputALine]{ch17/inputALine.cc}
            \end{itemize}
    \end{itemize}

    \only<2>{
        \begin{itemize}
        \item 图示:输入 \lstinline|hi c++| 后回车的效果
        \end{itemize}

        \begin{tikzpicture}
        [nodes in empty cells, nodes={minimum width=1.2cm, minimum height=.7cm}, row sep=-\pgflinewidth, column sep=-\pgflinewidth]
        \matrix(A) [matrix of nodes, ampersand replacement=\&, row 1/.style={nodes={draw=none}}, nodes={draw, anchor=center}]{
            \lstinline|s[0]| \& \lstinline|s[1]| \& \lstinline|s[2]| \& \lstinline|s[3]| \& \lstinline|s[4]| \& \lstinline|s[5]| \& \lstinline|s[6]| \& \lstinline|s[7]| \& \lstinline|...| \\
            \lstinline|h|   \& \lstinline|i|   \& \lstinline|\\0|   \& \lstinline|  |   \& \lstinline|  |   \& \lstinline|  |   \& \lstinline|  |   \& \lstinline|  |   \& \lstinline|  | \\
        };
        \end{tikzpicture}
    }
    \only<4>{
        \begin{itemize}
        \item 图示:输入 \lstinline|hi c++| 后回车的效果
        \end{itemize}

        \begin{tikzpicture}
        [nodes in empty cells, nodes={minimum width=1.2cm, minimum height=.7cm}, row sep=-\pgflinewidth, column sep=-\pgflinewidth]
        \matrix(A) [matrix of nodes, ampersand replacement=\&, row 1/.style={nodes={draw=none}}, nodes={draw, anchor=center}]{
            \lstinline|s[0]| \& \lstinline|s[1]| \& \lstinline|s[2]| \& \lstinline|s[3]| \& \lstinline|s[4]| \& \lstinline|s[5]| \& \lstinline|s[6]| \& \lstinline|s[7]|  \& \lstinline|...| \\
            \lstinline|h|   \& \lstinline|i|   \& \lstinline|  |   \& \lstinline|c|   \& \lstinline|+|   \& \lstinline|+|   \& \lstinline|\\n|   \& \lstinline|\\0|   \& \lstinline|  | \\
        };
        \end{tikzpicture}
    }
\end{frame}
%------------------------------------------------------------

%------------------------------------------------------------
\begin{frame}[fragile]
    \frametitle{C 风格字符串}

    \begin{itemize}
        \item<1-> 输出整个字符串
        \lstinputlisting[basicstyle=\ttfamily\scriptsize,language=C++,name=output]{ch17/output.cc}

        \item<2-> 访问字符
        \lstinputlisting[basicstyle=\ttfamily\scriptsize,language=C++,name=elementAccess]{ch17/elementAccess.cc}
            
    \end{itemize}
\end{frame}
%------------------------------------------------------------

\section{C 风格字符串的常用函数}

%------------------------------------------------------------
\begin{frame}[fragile]
    \frametitle{C 风格字符串的的常用函数}

    \begin{itemize}
        \item 头文件 \textbf{\lstinline|cstring|} 中有一些常用的函数
        \item \textbf{\lstinline|strlen(s)|} 函数
            \begin{itemize}
                \item 返回一个整数,表示字符串 $s$ 的\textbf{长度}
                \lstinputlisting[basicstyle=\ttfamily\scriptsize,language=C++,name=strlen]{ch17/strlen.cc}
            \end{itemize}
    \end{itemize}
\end{frame}
%------------------------------------------------------------

%------------------------------------------------------------
\begin{frame}[fragile]
    \frametitle{例 5.1:字符串转小写}

    \alt<2>{
        \lstinputlisting[basicstyle=\ttfamily\scriptsize,language=C++,name=A2a]{ch17/A2a.cc}
    }{
        \begin{exampleblock}{编程题}

            \begin{itemize}
                \item 输入一个长度不超过 $100$ 的字符串(只包含大小写字母),将其中的大写字母全部转为小写字母,其他字符不变,输出转变后的结果。

                \item 样例输入

                    \lstinline|HeLLO|

                \item 样例输出

                    \lstinline|hello|

            \end{itemize}

        \end{exampleblock}
    }
\end{frame}
%------------------------------------------------------------

%------------------------------------------------------------
\begin{frame}[fragile]
    \frametitle{C 风格字符串的的常用函数}

    \begin{itemize}
        \item<1-> \textbf{\lstinline|strcpy(s, t)|} 函数
            \begin{itemize}
                \item 无返回值,表示将字符串 $t$ \textbf{整体赋值}给字符串 $s$,字符串 $t$ 的内容保持不变
            \end{itemize}
        \item<4-> \textbf{\lstinline|strcat(s, t)|} 函数
            \begin{itemize}
                \item 无返回值,表示将字符串 $t$ \textbf{连接}到字符串 $s$ 之后,字符串 $t$ 的内容保持不变
            \end{itemize}
    \end{itemize}

    \only<2,5>{
        \begin{tikzpicture}
        [nodes in empty cells, nodes={minimum width=1.2cm, minimum height=.7cm}, row sep=-\pgflinewidth, column sep=-\pgflinewidth]
        \matrix(A) [matrix of nodes, ampersand replacement=\&, row 1/.style={nodes={draw=none}}, nodes={draw, anchor=center}]{
            \lstinline|s[0]| \& \lstinline|s[1]| \& \lstinline|s[2]| \& \lstinline|s[3]| \& \lstinline|s[4]| \& \lstinline|s[5]| \& \lstinline|s[6]| \& \lstinline|...| \\
            \lstinline|K|   \& \lstinline|O|   \& \lstinline|\\0|   \& \lstinline|  |   \& \lstinline|  |   \& \lstinline|  |     \& \lstinline|  |  \& \lstinline|  |\\
        };
        \end{tikzpicture}

        \begin{tikzpicture}
        [nodes in empty cells, nodes={minimum width=1.2cm, minimum height=.7cm}, row sep=-\pgflinewidth, column sep=-\pgflinewidth]
        \matrix(A) [matrix of nodes, ampersand replacement=\&, row 1/.style={nodes={draw=none}}, nodes={draw, anchor=center}]{
            \lstinline|t[0]| \& \lstinline|t[1]| \& \lstinline|t[2]| \& \lstinline|t[3]| \& \lstinline|t[4]| \& \lstinline|t[5]| \& \lstinline|t[6]| \& \lstinline|...| \\
            \lstinline|H|   \& \lstinline|a|   \& \lstinline|H|   \& \lstinline|a|   \& \lstinline|\\0|   \& \lstinline|  |     \& \lstinline|  |  \& \lstinline|  |\\
        };
        \end{tikzpicture}
    }
    \only<3>{
        \begin{tikzpicture}
        [nodes in empty cells, nodes={minimum width=1.2cm, minimum height=.7cm}, row sep=-\pgflinewidth, column sep=-\pgflinewidth]
        \matrix(A) [matrix of nodes, ampersand replacement=\&, row 1/.style={nodes={draw=none}}, nodes={draw, anchor=center}]{
            \lstinline|s[0]| \& \lstinline|s[1]| \& \lstinline|s[2]| \& \lstinline|s[3]| \& \lstinline|s[4]| \& \lstinline|s[5]| \& \lstinline|s[6]| \& \lstinline|...| \\
            \lstinline|H|   \& \lstinline|a|   \& \lstinline|H|   \& \lstinline|a|   \& \lstinline|\\0|   \& \lstinline|  |  \& \lstinline|  |   \& \lstinline|  | \\
        };
        \end{tikzpicture}

        \begin{tikzpicture}
        [nodes in empty cells, nodes={minimum width=1.2cm, minimum height=.7cm}, row sep=-\pgflinewidth, column sep=-\pgflinewidth]
        \matrix(A) [matrix of nodes, ampersand replacement=\&, row 1/.style={nodes={draw=none}}, nodes={draw, anchor=center}]{
            \lstinline|t[0]| \& \lstinline|t[1]| \& \lstinline|t[2]| \& \lstinline|t[3]| \& \lstinline|t[4]| \& \lstinline|t[5]| \& \lstinline|t[6]| \& \lstinline|...| \\
            \lstinline|H|   \& \lstinline|a|   \& \lstinline|H|   \& \lstinline|a|   \& \lstinline|\\0|   \& \lstinline|  |  \& \lstinline|  |   \& \lstinline|  | \\
        };
        \end{tikzpicture}
    }
    \only<6>{
        \begin{tikzpicture}
        [nodes in empty cells, nodes={minimum width=1.2cm, minimum height=.7cm}, row sep=-\pgflinewidth, column sep=-\pgflinewidth]
        \matrix(A) [matrix of nodes, ampersand replacement=\&, row 1/.style={nodes={draw=none}}, nodes={draw, anchor=center}]{
            \lstinline|s[0]| \& \lstinline|s[1]| \& \lstinline|s[2]| \& \lstinline|s[3]| \& \lstinline|s[4]| \& \lstinline|s[5]| \& \lstinline|s[6]| \& \lstinline|...| \\
            \lstinline|K|   \& \lstinline|O|   \& \lstinline|H|   \& \lstinline|a|   \& \lstinline|H|   \& \lstinline|a|   \& \lstinline|\\0| \& \lstinline|  |\\
        };
        \end{tikzpicture}

        \begin{tikzpicture}
        [nodes in empty cells, nodes={minimum width=1.2cm, minimum height=.7cm}, row sep=-\pgflinewidth, column sep=-\pgflinewidth]
        \matrix(A) [matrix of nodes, ampersand replacement=\&, row 1/.style={nodes={draw=none}}, nodes={draw, anchor=center}]{
            \lstinline|t[0]| \& \lstinline|t[1]| \& \lstinline|t[2]| \& \lstinline|t[3]| \& \lstinline|t[4]| \& \lstinline|t[5]| \& \lstinline|t[6]| \& \lstinline|...| \\
            \lstinline|H|   \& \lstinline|a|   \& \lstinline|H|   \& \lstinline|a|   \& \lstinline|\\0|   \& \lstinline|  |     \& \lstinline|  | \& \lstinline|  | \\
        };
        \end{tikzpicture}
    }
\end{frame}
%------------------------------------------------------------

%------------------------------------------------------------
\begin{frame}[fragile]
    \frametitle{讨论}

    \begin{block}{}
        \vspace{.5cm}
        \begin{center}
            {\Large 整数、小数都存在大小关系,\\那字符串存在大小关系吗?}
        \end{center}
        \vspace{.5cm}
    \end{block}
\end{frame}
%------------------------------------------------------------

%------------------------------------------------------------
\begin{frame}[fragile]
    \frametitle{字典序}

    \begin{itemize}[<+->]
        \item 字典序表示单词在字典中出现的先后顺序
        \begin{itemize}
            \item \lstinline|bear, apple, beer| 三个单词在字典中的先后顺序是什么?
            \item \lstinline|Bear, apple, Beer| 三个单词在字典中的先后顺序是什么?
        \end{itemize}
        \item 在计算机中会以 \textbf{\lstinline|ASCII| 码} 的数值大小为依据,比较每个字符
        \begin{itemize}
            \item \lstinline|Bear, 123, apple| 三个字符串的字典序大小是什么?
            \item \lstinline|Bee, Bee| 两个字符串的字典序大小是什么?
            \item \lstinline|Beer, Bee| 两个字符串的字典序大小是什么?
        \end{itemize}
    \end{itemize}

\end{frame}
%------------------------------------------------------------

%------------------------------------------------------------
\begin{frame}[fragile]
    \frametitle{字典序}

    \begin{itemize}
        \item<1-> \textbf{\lstinline|strcmp(s, t)|} 函数
        \begin{itemize}
            \item 返回值一个整数,表示字符串 $s, t$ 之间的字典序大小关系
            \item 返回值大于 $0$,表示字符串 $s$ 的字典序大于字符串 $t$
            \item 返回值等于 $0$,表示字符串 $s$ 的字典序等于字符串 $t$(两个字符串相同)
            \item 返回值小于 $0$,表示字符串 $s$ 的字典序小于字符串 $t$
        \end{itemize}
    \end{itemize}

\end{frame}
%------------------------------------------------------------

%------------------------------------------------------------
\begin{frame}[fragile]
    \frametitle{例 5.2:字典序比较}

    \alt<2>{
        \lstinputlisting[basicstyle=\ttfamily\scriptsize,language=C++,name=cmp]{ch17/cmp.cc}
    }{
        \begin{exampleblock}{编程题}

            \begin{itemize}
                \item 输入两个长度均不超过 $100$ 的字符串 $s, t$(均只包含大小写字母和数字字符),依据以下要求进行输出:
                \begin{itemize}
                    \item 如果 $s$ 的字典序大于 $t$,则输出 >
                    \item 如果 $s$ 的字典序等于 $t$,则输出 =
                    \item 如果 $s$ 的字典序小于 $t$,则输出 <
                \end{itemize}

                \item 样例输入

                    \lstinline|bee beer|

                \item 样例输出

                    \lstinline|<|

            \end{itemize}

        \end{exampleblock}
    }
\end{frame}
%------------------------------------------------------------

%------------------------------------------------------------
\begin{frame}[fragile]
    \frametitle{例 5.3:判断回文串}

    \alt<2>{
        \lstinputlisting[basicstyle=\ttfamily\scriptsize,language=C++,name=palindrome]{ch17/palindrome.cc}
    }{
        \begin{exampleblock}{编程题}

            \begin{itemize}
                \item 输入一个长度不超过 $100$ 的字符串 $s$(只包含大小写字母和数字字符)。若 $s$ 是回文串则输出 YES,否则输出 NO。\\
                    回文串表示字符串正着念与倒着念相同。

                \item 样例输入

                    \lstinline|level|

                \item 样例输出

                    \lstinline|YES|

            \end{itemize}

        \end{exampleblock}
    }
\end{frame}
%------------------------------------------------------------

%------------------------------------------------------------
\begin{frame}[fragile]
    \frametitle{例 5.4:输出子串}

    \alt<2>{
        \lstinputlisting[basicstyle=\ttfamily\scriptsize,language=C++,name=substr]{ch17/substr.cc}
    }{
        \begin{exampleblock}{编程题}

            \begin{itemize}
                \item 输入一个长度不超过 $100$ 的字符串 $s$(只包含大小写字母)。再输入两个整数 $pos, len$,请输出 $s$ 中从下标 $pos$ 开始、长度为 $len$ 的子串(保证不会越界)。

                \item 样例输入

                    \lstinline|abcdef|\\
                    \lstinline|2 3|

                \item 样例输出

                    \lstinline|cde|

            \end{itemize}

        \end{exampleblock}
    }
\end{frame}
%------------------------------------------------------------

\section{总结}

%------------------------------------------------------------
\begin{frame}[fragile]
    \frametitle{C 风格字符串}

    \begin{itemize}
        \item<1-> C 风格字符串
        
            \begin{itemize}
                \item 声明、初始化、输入、输出
            \end{itemize}

        \item<2-> 常用函数
        
            \begin{itemize}
                \item \lstinline|strlen(s)|:获取长度
                \item \lstinline|strcpy(s, t)|:整体赋值
                \item \lstinline|strcat(s, t)|:拼接
                \item \lstinline|strcmp(s, t)|:比较字典序
            \end{itemize}
       
    \end{itemize}
\end{frame}
%------------------------------------------------------------

%------------------------------------------------------------
\begin{frame}
    \begin{center}
        {\Huge Thank you!}
    \end{center}
\end{frame}
%------------------------------------------------------------

\end{document}
